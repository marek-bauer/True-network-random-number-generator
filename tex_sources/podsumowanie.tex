\chapter{Podsumowanie}
\thispagestyle{chapterBeginStyle}
Celem pracy było stworzenie konceptu generatora ''prawdziwych'' liczb losowych opartego o entropię pochodzącą z czasów dostępu do zasobów sieciowych oraz zaimplementowanie stworzonego narzędzia w celu wykazania jego poprawności. Wektor liczb będący wynikiem pracy takiego generatora musi spełniać warunek jednostajności oraz niezależności, zatem dodatkowym celem było przygotowanie narzędzi umożliwiających potwierdzenie poprawności rozwiązania poprzez przetestowanie wyników generatora pod tym kątem.\par
Początek pracy omawia czy możliwe jest stworzenie generatora prawdziwych liczb losowych, a który to problem można sprowadzić do pytania na temat determinizmu świata. Kolejna sekcja pochyla się nad analizą dwóch podejść do generowania liczb losowych. Jednym z nich było generowanie liczb na podstawie entropii świata zewnętrznego, natomiast drugim - koncepcja odrzucenia wymogu niezależności na rzecz generowania liczb pseudolosowych. Kolejno zaprezentowany został pomysł na źródło entropii w postaci czasów dostępów do zasobów sieciowych (takich jak czas odpowiedzi na ping) wraz z przesłankami świadczącymi o jego poprawności, po czym przeanalizowano teoretyczne problemy takiego podejścia. Czasy pingów mają specyficzny rozkład, w którym występuje prawo Benforda. Oznacza to, iż prawdopodobieństwo otrzymania jedynki na $k$-tym bicie nie jest stałe i jest zależne od długości liczby. Innym przeanalizowanym problemem była maksymalna precyzja pomiarów czasu, której możemy się spodziewać na współczesnych komputerach oraz jej wpływ na wydajność. Następny rozdział skupiał się na analizie i rozwiązaniu problemów technicznych, które rodził nasz generator. Został w nim omówiony pomysł na niwelowanie skutków niejednostajnego rozkładu bitów poprzez odrzucenie bitów o nazbyt odchylonym rozkładzie. Przedstawiono również kryteria rozkładu bitów akceptowalnych dla naszego generatora oraz zaprezentowano szczegółowy pseudokod generatora wraz z jego implementacją w języku Python i jej omówieniem. Dokonana została  również analiza różnych protokołów sieciowych, które można wykorzystać w celu pozyskania czasów dostępów do zasobów sieciowych. Końcem rozdziału przeprowadzono analizę wydajności już napisanego generatora - zarówno teoretyczną, jak i w środowisku testowym. Ostatni rozdział skupia się na problemie pokazania jednostajności oraz niezależności wektora liczb losowych. Zaczyna się on od wyjaśnienia, dlaczego nie może istnieć matematyczny dowód takich własności, a następnie proponuje testy statystyczne jako alternatywę. Ich idea została dokładanie omówiona uwzględniając problemy jakie rodzą. Następną sekcję poświęcono analizie istniejących już narzędzi testowania losowości oraz wyborze trzech generatorów różnej klasy, z którymi zostanie porównane nasze rozwiązanie. Wyjaśniono następnie, iż gotowe narzędzia nie posiadają satysfakcjonującego uzasadnienia swojej poprawności, przez co przygotowanych 10 testów inspirowanych istniejącymi rozwiązaniami, wzbogaconych o matematyczne uzasadnienie swojej poprawności. Rozdział kończy się zaprezentowaniem wyników przeprowadzonych testów - zarówno tych stworzonych na potrzebę pracy, jak i tych już istniejących.\par
Niniejsza praca wykazała możliwość stworzenia generatora ''prawdziwych'' liczb losowych opartego o czasy dostępów do zasobów sieciowych. Generowany przez niego ciąg liczb spełnia wymogi jednostajności i niezależności nie gorzej niż generatory rekomendowane do zastosowań kryptograficznych. Oznacza to, że teoretycznie nasz generator również mógłby być użyty w miejscach wymagających generowania losowych kluczy bez uszczerbku dla bezpieczeństwa tych systemów. Problemem, którego nie udało się w niniejszej pracy rozwiązać (i który najprawdopodobniej niemożliwy jest do rozwiązania) pozostaje wydajność tego rozwiązania. Spowodowany jest on wykorzystaniem mocno ograniczonego zasobu, jakim jest dostęp do sieci. Z tego powodu, generator ten najpewniej nie znajdzie zastosowania w przemyśle i pozostanie konceptem czystko akademickim. Może on jednak posłużyć jako generator wartości początkowych (seedu) dla generatorów liczb pseudolosowych lub jako dodatkowe źródło entropii dla innych rozwiązań. \par
Dalszych badań wymaga problem znaczącego spadku wydajności w przypadku równoległego pingowana w środowisku testowym. Koniecznym okazuje się także odpowiedź na pytanie, czy problem dotyczy tego jednego środowiska, czy jest szerzej reprodukowany oraz jakie są przyczyny gubienia pakietów, mechanizm wyboru bitów o akceptowanym rozkładzie można by również udoskonalić o użycie bitów o nieakceptowanych rozkładach w celu przekształcenia ich pewnej skończonej ilości w bit o rozkładzie akceptowanym dla naszego generatora, co pozwoliłoby w nieznaczny sposób podnieść wydajność systemu. Innym potencjalnym sposobem na jej podniesienie jest analiza, czy zawartość otrzymywanych pakietów może służyć jako dodatkowe źródło entropii. Z punktu widzenia zastosowań kryptograficznych, najważniejsze jest jednak sprawdzenie, jakie są możliwości manipulowania otrzymanymi czasami dostępów do zasobów sieciowych. Nurtującymi pozostaje pytanie. Czy przeprowadzanie ataku z wykorzystaniem fałszywej bramki domyślnej pozwoliłoby by na manipulowanie wynikami generatora? Czy jednak ze względu na odrzucenie najbardziej znaczących bitów operacja ta jest z praktycznego punktu widzenia niemożliwa do przeprowadzenia?




